\section{Introduction}

%The Internet has crossed new frontiers with access to it getting faster and cheaper. New applications and services are being offered. Information and communication is of utmost importance during emergency response. The Internet plays a crucial role in enabling access to critical information that can save lives of people. Internet is now seen as critical infrastructure � enabling remote health care, education, employment, e-governance, digital economy, and social networks. 

%In the reality of today�s Internet, the vision of digital inclusion faces the challenge of a growing digital divide, i.e., a growing disparity between those with sufficient access to the Internet and those who cannot afford the access to universal services. 

The Internet has evolved into a critical infrastructure for education, employment, e-governance, remote health care, digital economy, and social media. However, Internet today is facing the challenge of a growing digital divide, i.e., an increasing disparity between those with and without Internet access. Access problems often stem from sparsely spread populations living in physically remote locations, since it is simply not cost-effective for Internet Service Providers (ISPs) to deploy the required infrastructure for broadband Internet access in these areas. Coupled with physical limitations of terrestrial infrastructures (mainly due to distance) to provide last mile access, remote communities also incur higher costs for connection between the exchange and backbone network when using wired technologies, because the distances are longer. A large exchange may accommodate many users and allow for competition between service operators; in contrast, a rural/remote broadband connection does not usually offer economies of scale, increasing the costs per user. 

This problem is widely and publicly recognized. For example, in 2012 9.1 million homes in Europe still did not have fixed broadband coverage, more than 90\% of which are in rural areas \cite{BROAD2012}. Achieving ubiquitous mobile broadband coverage is also currently seen as not feasible by major operators as direct investment in local infrastructure may be uneconomic.

Addressing digital exclusion due to socio-economic barriers is also important. The United Nations revealed the global disparity in fixed broadband access, showing that access to fixed broadband in some countries costs almost 40 times their national average income \cite{FILD2010}. This problem is also applicable to developed countries where many individuals find themselves unable to pass a necessary credit check or living in circumstances that are too unstable to commit to lengthy broadband contracts \cite{LCDNet}. A recent survey in Nottingham, UK \cite{NCC2012} revealed that affordability is cited as the primary barrier, explicitly so by over 22.7\% of digitally excluded in the age of 16-44 years. 

We see enabling benevolence in the Internet (act of sharing resources) as a potential solution to solve the problem of digital exclusion caused due to socio-economic barriers. Lowest Cost Denominator Networking (LCDNet) \cite{LCDNet} is a new Internet paradigm that architects multi-layer resource pooling Internet technologies to support new low-cost access methods that could greatly reduce a network operator's direct investment in local infrastructure to support wider Internet access. LCDNet proposes to bring together several existing resource pooling Internet technologies to ensure that donors (users and network operators), who share their resources, are not affected and at the same time are incentivized for sharing their resources.  

Public Access WiFi Service (PAWS) \cite{PAWS} is based on LCDNet using a set of techniques that make use of the available unused capacity in home broadband networks and allowing Less-than-Best Effort (LBE) access to these resources \cite{LCDNet}. PAWS adopts an approach of community-wide participation, where broadband customers are enabled to donate controlled but free use of their high-speed broadband Internet by fellow citizens. Other initiatives have already explored sharing a user's broadband Internet connection via wireless (e.g., FON \cite{FON}). Although these methods are gaining worldwide acceptance, they are usually viewed as an extension of a user's paid service which is accessible only by other customers of the same service. In contrast, PAWS offers free access to essential services to all. 

To protect the consumer's paid service and the service provider revenue, it is essential to ensure that the guest user traffic does not impact perceived performance of the bandwidth donor (customer). The PAWS service is therefore constrained to offer a LBE access to network resources (lower quality compared to the standard Internet service offered to paying users). Various methods are being considered, including enabling LBE QoS (both in layers 2 and 3) in the network.

\begin{figure}[b]
\begin{center}
\includegraphics[width=1\linewidth]{paws.pdf}  
\caption{PAWS network.}
\label{fig:paws}
\end{center}
\end{figure}

PAWS is currently under deployment with 20 custom-made PAWS routers placed in a deprived community in Nottingham and another 10 routers in rural Scotland (Fig. \ref{fig:paws}). The testbed currently serves as a small-scale open network measurement observatory in the UK that will allow researchers to gather data about network availability, reachability, topology, security, and broadband performance from distributed vantage points in socio-economically deprived urban and rural areas. A significant contribution of the PAWS deployment is that it currently serves as a crowd-shared access network (like FON) for under-privileged users in urban and rural communities. This provides the research community with a wealth of information on the needs of under-privileged users in terms of their access patterns and what do they use Internet access for. The testbed also provides researchers the opportunity to understand behavioral patterns of home broadband users in terms of how do they share their home broadband networks with the public, e.g., how often do they switch off their home access points and when (day/time) do they switch it off.

However, PAWS has faced ongoing deployment challenges such as limited coverage and most importantly due to home user sharing patterns. We have noticed that the PAWS routers were always not available mainly due to home users plugging off the PAWS router from the Ethernet socket of the home router and reusing the socket for their own use or because they did not want to share their Internet connection with the others for certain periods of the day or for other reasons such as economic constraints placed on home users in underprivileged areas where they are forced to conserve energy by turning off the routers at nights. Figure \ref{fig:paws-avail} presents a six-month view of the PAWS routers status (available/unavailable) logs demonstrating that not a single router was available continuously over the entire duration. These observed user behaviors have become serious challenges for the successful adoption of PAWS. 

\begin{figure}[h]
\begin{center}
\includegraphics[width=1\linewidth]{paws-avail.pdf}  
\caption{PAWS routers availability.}
\label{fig:paws-avail}
\end{center}
\end{figure}

The underlying problem with PAWS or any crowd-shared network (such as FON) is that they serve as single point of Internet access to users within the coverage of the wireless router and hence have no provision to extend the coverage or to provide any redundancy during unavailability of the routers which is mainly due to the sharer�s sharing tendencies or policies. A potential solution to these problems would be to extend the PAWS network as a crowd-shared mesh network. Such a network would allow home broadband users to share part of their own broadband connection to the public for free while also connected to each other as a wireless mesh providing extended coverage. This also offers network redundancy to the Internet backhaul even when some sharers decide not to share their backhaul Internet connection for certain periods of time. 

In this paper, we explore the potential benefits of enabling PAWS or any crowd-shared wireless network as a crowd-shared wireless mesh network. Our paper provides the following contributions:
%\begin{enumerate}
%\end{enumerate}

The rest of the paper is structured as follows: In Section \ref{sec:architecture}, we discuss the architecture of the simulated mesh network. Section \ref{sec:methodology} presents the simulation methodology. In Section \ref{sec:evaluation}, we evaluate the benefits using simulations and discuss the results, while in Section \ref{sec:discussion} we provide a discussion. Finally we conclude in Section \ref{sec:conclusion}.
