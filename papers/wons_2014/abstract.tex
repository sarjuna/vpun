Universal access to Internet is crucial, and as such, there have been several initiatives to enable wider access to the Internet. Public Access WiFi Service (PAWS) is one such initiative that takes advantage of the the available unused capacity in home broadband connections and allows Less-than-Best Effort (LBE) access to these resources, as exemplified by Lowest Cost Denominator Networking (LCDNet). PAWS has been recently deployed in a deprived community in Nottingham, and, as any crowd-shared network, it faces limited coverage, since there is single point of Internet access per guest user, whose availability depends on user sharing policies.

To mitigate this problem and extend the coverage, we consider a crowd-shared wireless mesh network (WMN) in which the home routers are interconnected as a mesh. Such a network provides multiple points of Internet access and can enable resource pooling across all available paths to the Internet backhaul. In this paper, we investigate the potential benefits of a crowd-shared WMN for public Internet access by performing a comparative study between such a network and PAWS, using simulations. To this end, we present an algorithm that selects the gateway and the shortest path for guest user traffic redirection through the WMN. Our simulations are driven from real user sharing patterns, collected from the PAWS deployment in Nottingham. Our simulation results show that a crowd-shared WMN can provide much higher utilization of the shared bandwidth and can accommodate a substantially larger volume of guest user traffic. We further investigate the trade-off between shared bandwidth utilization and the maximum number of hops between the local home router and the gateway.

