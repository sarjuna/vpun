\section{Conclusions and Future Work}
\label{sec:conclusion}

In this paper, we quantified the benefits of extending the coverage of any crowd-shared mesh network (e.g., PAWS) by connecting the home routers as a mesh. A crowd-shared WMN can mitigate the fundamental problem of any crowd-shared network, i.e., the presence of a single point of access for each guest user. In a crowd-shared WMN, the path redundancy towards the Internet backhaul can be exploited to achieve better resource utilization, especially during periods of limited Internet access availability. According to our trace-driven simulations, a crowd-shared WMN can offer much better utilization of the shared bandwidth,  expanding free Internet access to a larger number of users.

In this paper, SDN and the notion of Virtual Public Networks \cite{EWSDN} constitute the underlying assumptions for the configuration and management of such WMNs by third-party virtual network operators. Although VPuN can facilitate the federation of open wireless home networks and allow their management by a single operator, crowd-shared WMN configuration and management is certainly not a trivial task. For example, decisions for guest user traffic redirections require information about the WMN and access link utilization as well as home user sharing policies. A large WMN may require routing traffic across multiple hops, which, in turn, raises the need for the coordination of flow table updates (e.g., using OpenFlow \cite{OPENFLOW}) in order to avoid packet loss and service disruptions. 

As part of future work, we plan to gain more insights into these problems and investigate solutions by implementing and deploying a SDN control plane in an experimental WMN.  
